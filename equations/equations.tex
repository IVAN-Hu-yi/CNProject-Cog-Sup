\documentclass{article}
\usepackage{amsmath}
\usepackage{amssymb}
\usepackage{bm}

\begin{document}

\title{Hodgkin-Huxley Neuron Model Equations (with Extensions)}
\author{}
\date{}
\maketitle

\section{Hodgkin-Huxley Neuron Model}
The voltage dynamics for a single-compartment neuron is described by:
\begin{align}
  C_m \frac{dV}{dt} = - I_\mathrm{L} - \sum_{x \in \mathbb{X}} I_x  + I_\mathrm{ext} + I_\mathrm{noise} 
\end{align}

\noindent where $C_m$ is the membrane capacitance, $I_\mathrm{ext}$ is external current, and $I_\mathrm{noise}$ is noise current. Here $I_x$ is the ionic current for each ion channel type $x$ in the set $\mathbb{X}$, which includes sodium (Na), potassium (K), L-type and T-type calcium (CaL, CaT), hyperpolarization-activated (H), slow potassium (Ks), afterhyperpolarization (AHP), and calcium-activated non-specific (CAN) channels (i.e., $\mathbb{X} = \{Na, K, CaL, CaT, H, Ks, AHP, CAN\}$). Each ionic current follows Ohm's with activation and inactivation dynamics:

\begin{align}
  I_x = g_x m^a h^b  (V - V_x)
\end{align}
\noindent where $g_x$ is the maximal conductances for ion type $x$, $m$ and $h$ are activation and inactivation gating variables, respectively, and $V_x$ is the reversal potential for ion type $x$. The parameters $a$ and $b$ depend on the specific channel type describing the on/off kinetics of the channel provided in the table. All gating variables evolve according to first-order kinetics:

\begin{align}
  \tau_m \frac{dx}{dt} = (m_\infty - m)
\end{align}

\noindent Here the steady-state value $m_\infty$ is a function of the membrane potential $V$, and $\tau_m$ is the voltage-dependent time constant for the gating variable $m$. The same applies to inactivation variables $h$.

\section{Ionic Currents}

Belows are equations for various ionic currents in the neuron model:
\begin{align}
I_\mathrm{L} = g_\mathrm{L}(V - V_\mathrm{L})
\end{align}

\begin{align}
I_\mathrm{Na} = g_\mathrm{Na} \, m_\mathrm{Na}^3 \, h_\mathrm{Na} \, (V - V_\mathrm{Na})
\end{align}

\begin{align}
I_\mathrm{K} = g_\mathrm{K} \, n_\mathrm{K}^4 \, (V - V_\mathrm{K})
\end{align}

\begin{align}
I_\mathrm{CaL} = g_\mathrm{CaL} \, m_\mathrm{CaL}^2 (V - V_\mathrm{CaL})
\end{align}

\begin{align}
I_\mathrm{CaT} = g_\mathrm{CaT} \, (m_\mathrm{CaT}^{\infty})^2 \, h_\mathrm{CaT} (V - V_\mathrm{CaT})
\end{align}

\begin{align}
I_\mathrm{AHP} = g_\mathrm{AHP} \, x_\mathrm{AHP} (V - V_\mathrm{AHP})
\end{align}

\begin{align}
I_\mathrm{CAN} = g_\mathrm{CAN} \, x_\mathrm{CAN} (V - V_\mathrm{CAN})
\end{align}

\begin{align}
I_\mathrm{Ks} = g_\mathrm{Ks} \, m_\mathrm{Ks} \, h_\mathrm{Ks} (V - V_\mathrm{Ks})
\end{align}

\begin{align}
I_\mathrm{H} = g_\mathrm{H} \, m_\mathrm{H} \, (V - V_\mathrm{H})
\end{align}

\section*{Gating and Activation Variables}

\subsection*{Sodium Channel (Na)}
\begin{equation}
m_{Na}(V) = \frac{1}{1 + \exp\left(-\frac{V + 30}{9.5}\right)}
\end{equation}

\begin{equation}
h_{Na,\infty}(V) = \frac{1}{1 + \exp\left(\frac{V + 53}{7}\right)}
\end{equation}

\begin{equation}
\tau_h(V) = 0.37 + 2.78 \cdot \frac{1}{1 + \exp\left(\frac{V + 40.5}{6}\right)}
\end{equation}

\subsection*{Potassium Channel (K)}
\begin{equation}
n_{K,\infty}(V) = \frac{1}{1 + \exp\left(-\frac{V + 30}{10}\right)}
\end{equation}

\begin{equation}
\tau_n(V) = 0.37 + 1.85 \cdot \frac{1}{1 + \exp\left(\frac{V + 27}{15}\right)}
\end{equation}

\subsection*{L-type Calcium Channel (CaL)}
\begin{equation}
m_{CaL,\infty}(V) = \frac{1}{1 + \exp\left(-\frac{V + 12}{7}\right)}
\end{equation}

\begin{equation}
\tau_{CaL}(V) = 10^{(0.6 - 0.02 \cdot V)}
\end{equation}

\subsection*{T-type Calcium Channel (CaT)}
\begin{equation}
m_{CaT,\infty}(V) = \frac{1}{1 + \exp\left(-\frac{V + 57}{6.2}\right)}
\end{equation}

\begin{equation}
h_{CaT,\infty}(V) = \frac{1}{1 + \exp\left(\frac{V + 81}{4}\right)}
\end{equation}

\begin{equation}
\tau_{mT}(V) = 0.612 + \frac{1}{\exp\left(\frac{V + 132}{-16.7}\right) + \exp\left(\frac{V + 16.8}{18.2}\right)}
\end{equation}

\begin{equation}
\tau_{hT}(V) = 
\begin{cases} 
\exp\left(\frac{V + 467}{66.6}\right) & \text{if } V < -80 \\
\exp\left(-\frac{V + 22}{10.5}\right) + 28 & \text{otherwise}
\end{cases}
\end{equation}

\subsection*{Hyperpolarization-Activated Channel (H)}
\begin{equation}
m_{H,\infty}(V) = \frac{1}{1 + \exp\left(\frac{V - V_{\tau,peak}}{k_{\tau}}\right)}
\end{equation}

\begin{equation}
\tau_{mH}(V) = \tau_{min} + \frac{\tau_{diff}}{\exp\left(\frac{V - V_{\tau,peak}}{k_{\tau}}\right) + \exp\left(-\frac{V - V_{\tau,peak}}{k_{\tau}}\right)}
\end{equation}

\subsection*{CAN Channel}
\begin{equation}
x_{CAN,\infty}([Ca^{2+}]) = \frac{a_{CAN} \cdot [Ca^{2+}]}{a_{CAN} \cdot [Ca^{2+}] + b_{CAN}}
\end{equation}

\begin{equation}
\tau_{xCAN}([Ca^{2+}]) = \frac{1}{a_{CAN} \cdot [Ca^{2+}] + b_{CAN}}
\end{equation}

\subsection*{AHP Channel}
\begin{equation}
x_{AHP,\infty}([Ca^{2+}]) = \frac{a_{AHP} \cdot [Ca^{2+}]}{a_{AHP} \cdot [Ca^{2+}] + b_{AHP}}
\end{equation}

\begin{equation}
\tau_{xAHP}([Ca^{2+}]) = \frac{1}{a_{AHP} \cdot [Ca^{2+}] + b_{AHP}}
\end{equation}

\subsection*{Slow Potassium Channel (Ks)}
\begin{equation}
m_{Ks,\infty}(V) = \frac{1}{1 + \exp\left(-\frac{V + 44}{5}\right)}
\end{equation}

\begin{equation}
h_{Ks,\infty}(V) = \frac{1}{1 + \exp\left(\frac{V + 74}{9.3}\right)}
\end{equation}

\begin{equation}
\tau_{hKs}(V) = 200 + \frac{4800}{1 + \exp\left(-\frac{V + 50}{9.3}\right)}
\end{equation}

\subsection*{General Form}
All gating variables follow first-order kinetics:
\begin{equation}
\tau_x \frac{dx}{dt} = (x_{\infty} - x)
\end{equation}
where $x$ represents the gating variable, $x_{\infty}$ is its steady-state value, and $\tau_x$ is its voltage-dependent or calcium-dependent time constant.

\end{document}
\vspace{2em}
\section{Intracellular Calcium Dynamics}
\begin{align}
[\mathrm{Ca}^{2+}](t+\Delta t) = [\mathrm{Ca}^{2+}](t) - \tau_\mathrm{Ca} \cdot \text{Geometric\_Factor} \cdot (I_\mathrm{CaL}(t) + I_\mathrm{CaT}(t))
\end{align}

\vspace{2em}
\section{Plasticity/Excitability Variable}
\begin{align}
w(t+\Delta t) = w(t) + \eta_w \left( \Theta([\mathrm{Ca}^{2+}](t) - \theta_\mathrm{Ca}) - w(t) \right)
\end{align}
where $\Theta(x)$ is the Heaviside step function.

\end{document}
